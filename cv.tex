\documentclass[
	a4paper,
	10pt,
	roman,
	colorlinks,
	linkcolor = purple,
	filecolor = purple,
	citecolor = blue,
	urlcolor = blue
]{moderncv}
\moderncvtheme[black]{classic}

\usepackage[margin = 10mm]{geometry}
\usepackage[english]{babel}
\usepackage[none]{hyphenat}
\usepackage[yyyymmdd]{datetime}
\usepackage[unicode, pdfpagelabels = false]{hyperref}
\usepackage{amsmath}
\usepackage{fontspec}
% overrides for ModernCV class
\usepackage{cv}

\name{Omri}{Bornstein}
% \photo[60pt]{assets/author.png}
\title{Software Engineer}
\quote{Go/Linux enthusiast, curiously seeking challenges \& professional development}
\email{omribor@gmail.com}
\homepage{applegamer22.github.io}
\social[linkedin]{omri-bornstein}
\social[github]{AppleGamer22}
\extrainfo{Updated on \textbf{\today}}

\raggedleft\raggedright\begin{document}
	\maketitle
	\section{Education}
	\cventry{2020 --- 2023}{Bachelor of Computer Science}{\href{https://www.monash.edu/it}{Monash University}}{Melbourne}{}{}
	% \cventry{2017 --- 2019}{SACE}{\href{https://asms.sa.edu.au}{Australian Science \& Mathematics School} (ASMS)}{Adelaide}{}{}

	\section{Skills}\label{sec:skills}
	\begin{itemize}
		\item \textbf{Programming Languages}: \textbf{Go}, JavaScript/TypeScript, Python, Kotlin/Java, C/C++
		\item \textbf{Tools}: \textbf{Git} (with GitHub/GitLab), MongoDB, \textbf{Docker}, \textbf{Terraform}, gRPC/Protocol Buffers, \href{https://gohugo.io}{Hugo}
		\item \textbf{Platforms}: \textbf{Linux}, \textbf{cloud-native}, web servers/browsers, macOS, Windows, Azure, Firebase
		\item \textbf{Design \& Implementation}: algorithms \& data structures, object-oriented programming, test-driven development
		% \item \textbf{Soft Skills}: technical writing, critical/analytical thinking, presenting/public speaking, research, troubleshooting/debugging, explaining, collaboration/teamwork
		\item \textbf{Hobbies}: \href{https://applegamer22.github.io/about/#hobbies}{playing guitar}, listening to/analysing music, self-directed learning, personal projects
	\end{itemize}

	\section{Experience}
	\subsection{\href{https://monsec.io/team/\#2023-team}{Monash Cyber Security Club} (MonSec)}
	\cventry{2023}{President}{}{}{\hyperref[sec:skills]{Terraform, Azure, Git, Hugo}}{\begin{itemize}
		% \item Coordinated collaboration with the university's \href{https://www.monash.edu/it}{Faculty of Information Technology} for purposes of events and advertising.
		\item Involved with industry relations for the purposes of sponsorship deals and collaborations.
		\item Overhauled the \href{https://monsec.io}{club's website} for greater usability.
		\item Assisted in adding more Azure infrastructure and an automation pipeline to the \href{https://ctf.monsec.io}{club's CTFd server} with \href{https://terraform.io}{Terraform} and \href{https://github.com/features/actions}{GitHub Actions}.
		% \item Successfully registered the club for \href{https://www.canva.com/canva-for-nonprofits/}{Canva} and \href{https://socialimpact.github.com}{GitHub} non-profit licenses.
		\item Organised and ran an introductory-level \href{https://youtu.be/WXTVsIHrAKM}{workshop} about command-line Linux.
	\end{itemize}}
	\cventry{2022 Q3---Q4}{Vice President}{}{}{}{\begin{itemize}
		\item Coordinated collaboration with the university's \href{https://www.monash.edu/it}{Faculty of IT} for purposes of events and advertising.
		% \item Involved with industry relations for the purposes of sponsorship deals and collaborations.
		% \item Changed our \href{https://ctf.monsec.io}{CTFd}'s \href{https://github.com/hmrserver/CTFd-theme-pixo}{theme}. In addition, I disabled the theme's default CRT-like flicker effect by editing its CSS files, such that people with epilepsy would not get an inferior experience.
		\item Organised and ran an introductory-level \href{https://youtu.be/oc_Ndi7p3Eg}{workshop} about \href{https://en.wikipedia.org/wiki/Steganography}{steganography}.
		% \item Represented the club at Faculty of IT's \href{https://www.monash.edu/it/events/2022/take-ctrl}{Take CTRL} and \href{https://www.monash.edu/it/events/2022/munch-and-mingle}{Munch \& Mingle}.
		% \item \href{https://en.wikipedia.org/wiki/Capture_the_flag_(cybersecurity)}{Capture the Flag} (CTF) participation: \href{https://applegamer22.github.io/posts/uactf/}{The University of Adelaide's CTF}, \href{https://applegamer22.github.io/posts/shell_ctf/}{SHELL CTF}, \href{https://applegamer22.github.io/posts/downunderctf/}{DownUnderCTF}
	\end{itemize}}
	\cventry{2022 Q1---Q2}{Secretary}{}{}{\hyperref[sec:skills]{Python, Git}}{\begin{itemize}
		\item Wrote a custom Jupyter notebook for membership base analysis and reporting purposes.
		% \item Wrote a \href{https://www.python.org}{Python} data visualisation program in order to analyse membership data, such that it would be easier for future committees to compile reports, to understand registration trends and to apply for student association grants.
		% \item Organised and recorded official committee and club meetings.
		% \item Represented the club during the orientation week of 2022's $1^{\text{st}}$ semester.
		\item Organised and ran an introductory-level about \href{https://youtu.be/893L13SxDUg}{workshop} binary reverse-engineering.
		\item Wrote a guide on the \href{https://monsec.io/resources/}{resources page of the club's website} on how to easily install and set-up a \href{https://www.kali.org}{Kali Linux} virtual machine.
	\end{itemize}}
	\cventry{2021 Q3---Q4}{Assistant Member Training Officer}{}{}{}{
		\begin{itemize}
			\item Helped to organise and ran a workshop about brute-forcing tools used for penetration testing.
			\item Assisted in the club's management and operations.
			% \item Participated in \href{https://applegamer22.github.io/posts/angstromctf/}{{\aa}ngstromCTF}.
		\end{itemize}
	}

	\subsection{Research}
	\hypertarget{FIT3144}{
		\cventry{2023}{Research Assistant}{Monash University's \href{https://handbook.monash.edu/2023/units/FIT3144}{FIT3144 unit}}{\hyperref[sec:skills]{JavaScript, Python \& Git}}{}{
			\href{https://github.com/HPI-ELEA/elea/pulls?q=author\%3AAppleGamer22}{Extended (available on GitHub)} a browser-based tool \href{https://dl.acm.org/doi/abs/10.1145/3583133.3590723}{(Wagner et al., 2023)} used for building evolutionary algorithms in educational settings. Supervised by \href{https://research.monash.edu/en/persons/markus-wagner}{Dr. Markus Wagner}.
		}
	}
	\hypertarget{FIT2082}{
		\cventry{2021 Q3---Q4}{Research Assistant}{Monash University's \href{https://handbook.monash.edu/2021/units/FIT2082}{FIT2082 unit}}{\hyperref[sec:skills]{C/C++, Python \& Linux}}{}{
			\begin{itemize}
				\item \href{https://github.com/AppleGamer22/FIT2082}{Contributed (available on GitHub)} to an \href{https://github.com/gkgange/lazycbs}{existing codebase}, based on prior research \href{https://ojs.aaai.org/index.php/ICAPS/article/view/3471}{(Gange, Harabor and Stuckey, 2021)} about \textsl{Lazy CBS}, a \href{https://en.wikipedia.org/wiki/Pathfinding\#Multi-agent_pathfinding}{Multi-Agent Path Finding} (MAPF) algorithm.
				% \item Worked on enabling \httplink[\textsl{Python}-to-\textsl{C++} bindings]{pybind11.readthedocs.io/en/stable/}, such that the compiled \textsl{Lazy CBS} codebase can be used as a Python-facing library for future projects.
				\item Built with C/C++ and \href{https://python.org}{Python} for Linux-based platforms. Supervised by \href{https://research.monash.edu/en/persons/daniel-harabor}{Dr. Daniel Harabor} and \href{https://research.monash.edu/en/persons/mor-vered}{Dr. Mor Vered}.
				% \begin{itemize}
					% \item Modified the \textsl{Lazy CBS} codebase such that the algorithm also outputs the final set of constraints that is used to rule out paths, such that \textsl{Lazy CBS} is formally an \textbf{Explainable} Multi-Agent Path Finding (XMAPF) algorithm.
				% \end{itemize}
			\end{itemize}
		}
	}

	\subsection{Freelancing}
	\cventry{2023 Q3---Q4}{Associate Software Engineer}{Radio Monash}{Clayton}{volunteering}{
		Advised the leadership team on software deployment and server migration of their audio stream.
	}
	\cventry{2021 Q3---Q4}{Software Engineer}{Contract}{Melbourne}{\hyperref[sec:skills]{Go \& test-driven development}}{
		Implemented a custom asynchronous fault-tolerant file back-up system that enables the continuation of file transferring from a variably-approximate point in time before the disruption. Available at \ghurl{AppleGamer22/rb}.
	}

	\section{\texorpdfstring{Projects \small \href{https://applegamer22.github.io/about/\#open-source-contributions}{external contributions}}{Projects}}
	\hypertarget{raker}{
		\cventry{since May 2022}{\texttt{raker}}{\ghurl{AppleGamer22/raker}}{\hyperref[sec:skills]{Go, Docker \& MongoDB}}{}{
			A social media scraper that is interfaced via a server-side rendered HTML user interface (or a CLI), and is managed by a REST API and a NoSQL database. I presented this project at \href{https://www.meetup.com/golang-mel/events/293777783/}{Melbourne's Go meet-up}.
		}
	}
	\hypertarget{stalk}{
		\cventry{since May 2022}{\texttt{stalk}}{\ghurl{AppleGamer22/stalk}}{\hyperref[sec:skills]{Go, Linux \& macOS}}{}{
			A cross-platform file-watcher that can run a command after each file-system operation on a given set of files or simply wait once until a file is changed.
		}
	}
	\hypertarget{cocainate}{
		\cventry{since January 2022}{\texttt{cocainate}}{\ghurl{AppleGamer22/cocainate}}{\hyperref[sec:skills]{Go, macOS \& Linux}}{}{
			A cross-platform re-implementation of the macOS utility \href{https://github.com/apple-oss-distributions/PowerManagement/tree/main/caffeinate}{\texttt{caffeinate}} that keeps the screen turned on either until stopped, for a set duration of time or while another process still runs.
		}
	}
	\hypertarget{sp}{
		\cventry{2020 --- 2021}{\texttt{sp}}{\ghurl{AppleGamer22/sp}}{\hyperref[sec:skills]{Kotlin/Java}}{}{
			A \href{https://papermc.io}{Minecraft server plugin} that enforces password authentication on player before allowing client-server interaction.
		}
	}
	\hypertarget{scr}{
		\cventry{2019 --- 2022}{scr-web}{\ghurl{AppleGamer22/scr-web}}{\hyperref[sec:skills]{TypeScript, Angular, Docker \& MongoDB}}{}{
			My previous attempt at building a full-stack (and a CLI) social media scraper with a single-page website framework and a RESTful server.
		}
	}
	\end{document}
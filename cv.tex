\documentclass[
	a4paper,
	10pt,
	roman,
	colorlinks,
	linkcolor = purple,
	filecolor = purple,
	citecolor = blue,
	urlcolor = blue
]{moderncv}
\moderncvtheme[black]{classic}

\usepackage[margin = 15mm]{geometry}
\usepackage[english]{babel}
\usepackage[none]{hyphenat}
\usepackage[yyyymmdd]{datetime}
\usepackage[unicode, pdfpagelabels = false]{hyperref}
\usepackage{amsmath}
\usepackage{fontspec}

\makeatletter
	\RenewDocumentCommand{\section}{sm}{
		\par\addvspace{1ex}
		\phantomsection{}
		\addcontentsline{toc}{section}{#2}
		\strut\sectionstyle{\textbf{\huge #2}}
		\par\nobreak\addvspace{0.5ex}{\@afterheading}
	}
\makeatother

\makeatletter
	\RenewDocumentCommand{\subsection}{sm}{
		\par\addvspace{1ex}
		\phantomsection{}
		\addcontentsline{toc}{subsection}{#2}
		\strut\subsectionstyle{\textbf{\large #2}}
		\par\nobreak\addvspace{0.5ex}{\@afterheading}
	}
\makeatother

\newcommand\ghurl[1]{\href{https://github.com/#1}{\nolinkurl{#1} on GitHub}}

\renewcommand{\dateseparator}{-}
\renewcommand{\labelitemi}{$\bullet$}
\renewcommand{\labelitemii}{$\circ$}

\setmonofont[
	Contextuals = Alternate,
	Scale = MatchLowercase,
	BoldFont = FiraCode-Medium.ttf,
	SlantedFont = FiraCode-Regular.ttf,
	SlantedFeatures = FakeSlant
]{Fira Code}
\setlength{\hintscolumnwidth}{0.15\textwidth}

\name{Omri}{Bornstein}
% \photo[64pt]{assets/author.png}
\title{Software Engineer}
\address{Greater Melbourne Area}{}{Australia}
\email{omribor@gmail.com}
\homepage{applegamer22.github.io}
\social[linkedin]{omri-bornstein}
\social[github]{AppleGamer22}
\extrainfo{Updated on \textbf{\today}}

\raggedleft\raggedright\begin{document}
	\maketitle
	\section{Education}
	\cventry{2020 \\ In Progress}{Bachelor of Computer Science}{\href{https://www.monash.edu/it}{Monash University}}{Melbourne}{}{}
	\cventry{2017 \\ 2019}{South Australian Certificate of Education}{\href{https://asms.sa.edu.au}{Australian Science \& Mathematics School} (ASMS)}{Adelaide}{}{}

	\section{Skills}
	\subsection{Computer Programming Languages}
	\cvitemwithcomment{\href{https://go.dev}{Go}}{
		\hyperlink{raker}{\texttt{raker}},
		\hyperlink{stalk}{\texttt{stalk}} \&
		\hyperlink{cocainate}{\texttt{cocainate}}
	}{server-side \& command-line interfaces (CLIs)}
	\cvitemwithcomment{\href{https://typescriptlang.org}{TypeScript}}{\hyperlink{scr}{scr-cli \& scr-web}}{full-stack}
	% \cvitemwithcomment{JavaScript}{}{client-side on web browsers}
	\cvitemwithcomment{\href{https://python.org}{Python}}{\hyperlink{FIT2082}{FIT2082 research project}}{data analysis}
	\cvitemwithcomment{\href{https://kotlinlang.org}{Kotlin}/Java}{\hyperlink{sp}{\texttt{sp}}}{Minecraft plugins}
	\cvitemwithcomment{C/C++}{\hyperlink{FIT2082}{FIT2082 research project}}{}
	\subsection{Document Markup Languages}
	\cvitemwithcomment{HTML \& CSS}{}{client-side UI on web browsers}
	\cvitemwithcomment{\TeX/\LaTeX}{}{PDF document typesetting}
	\cvitemwithcomment{Markdown}{}{technical documentation and communication}
	% \subsection{Natural Languages}
	% \cvitemwithcomment{English}{full professional proficiency}{}
	% \cvitemwithcomment{Hebrew}{native or bilingual proficiency}{}
	\subsection{Tools}
	\cvitemwithcomment{\href{https://git-scm.com}{Git}}{\hyperref[sec:open-source]{open-source projects}}{source code version control}
	\cvitemwithcomment{\href{https://github.com}{GitHub}/\href{https://gitlab.com}{GitLab}}{\hyperref[sec:open-source]{open-source projects}}{collaboration \& CI/CD}
	\cvitemwithcomment{\href{https://www.mongodb.com}{MongoDB}}{
		\hyperlink{raker}{\texttt{raker}} \& \hyperlink{scr}{scr-cli \& scr-web}
	}{document non-relational database}
	\cvitemwithcomment{SQL}{}{relational database querying}
	\cvitemwithcomment{\href{https://www.docker.com}{Docker}}{
		\hyperlink{raker}{\texttt{raker}} \& \hyperlink{scr}{scr-cli \& scr-web}
	}{container-style packaging}
	\cvitemwithcomment{\href{https://kubernetes.io}{Kubernetes}}{}{container orchestration}
	\cvitemwithcomment{\href{https://vagrantup.com}{Vagrant}}{}{virtual machine (VM) management}
	% \cvitemwithcomment{\href{https://terraform.io}{Terraform}}{}{infrustracture definitions as code}

	\subsection{Other}
	\begin{itemize}
		\item \textbf{Platforms}: Linux, Cloud Native, web servers/browsers, macOS, Windows
		\item \textbf{Soft Skills}: technical writing, critical/analytical thinking, presenting/public speaking, research, troubleshooting/debugging, explaining, collaboration/teamwork
	\end{itemize}

	\section{Projects}
	\subsection{Open-Source}\label{sec:open-source}
	\hypertarget{raker}{
		\cventry{May 2022 \\ Present}{\texttt{raker}}{\ghurl{AppleGamer22/raker}}{}{}{
			\begin{itemize}
				\item A social media scraper that is interfaced via a server-side rendered HTML user interface (or a CLI), and is managed by a REST API and a NoSQL database.
				\item Server-side is built with \href{https://go.dev}{Go}, \href{https://www.mongodb.com}{MongoDB}, \href{https://jwt.io}{JSON Web Tokens} (JWTs) and \href{https://www.docker.com}{Docker}.
				\item Client-side is built with HTML/CSS (\href{https://getbootstrap.com}{Bootstrap}).
				\item The companion CLI utility and configuration are built with \href{https://cobra.dev}{Cobra} and \href{https://github.com/spf13/viper}{Viper}.
			\end{itemize}
		}
	}
	\hypertarget{stalk}{
		\cventry{May 2022 \\ Present}{\texttt{stalk}}{\ghurl{AppleGamer22/stalk}}{}{}{
			\begin{itemize}
				\item A cross-platform file-watcher that can run a command after each file-system operation on a given files or simply wait once until a file is changed.
				\item Built with \href{https://go.dev}{Go}, \href{https://cobra.dev}{Cobra} and \href{https://github.com/fsnotify/fsnotify}{FSnotify}.
			\end{itemize}
		}
	}
	\hypertarget{cocainate}{
		\cventry{January 2022 \\ Present}{\texttt{cocainate}}{\ghurl{AppleGamer22/cocainate}}{}{}{
			\begin{itemize}
				\item A cross-platform re-implementation of the macOS utility \href{https://github.com/apple-oss-distributions/PowerManagement/tree/main/caffeinate}{\texttt{caffeinate}} that keeps the screen turned on either until stopped, for a set duration of time or while another process still runs.
				\item Built with \href{https://go.dev}{Go} and \href{https://cobra.dev}{Cobra}.
			\end{itemize}
		}
	}
	\cventry{December 2021}{CTFtime Discord Bot}{\ghurl{monsec/ctftime-discord-bot}}{}{}{
		\begin{itemize}
			\item A Discord bot for \href{https://monsec.io/contact/}{MonSec}'s Discord server, that fetches statistics about competing \href{https://en.wikipedia.org/wiki/Capture_the_flag_(cybersecurity)}{Capture the Flag} (CTF) teams from \href{https://ctftime.org}{CTFtime}, and displays them in the Discord interface.
			\item Built with \href{https://go.dev}{Go}.
		\end{itemize}
	}
	\hypertarget{sp}{
		\cventry{June 2020 \\ January 2021}{\texttt{sp}}{\ghurl{AppleGamer22/sp}}{}{}{
			\begin{itemize}
				\item My first attempt at building a \href{https://papermc.io}{Minecraft server plugin}. This plugin adds the requirement that the player supplies the password (via a server command) before proper server interaction is allowed, and as long as the password isn't provided, the currently-unauthorized player is blinded and immobile.
				\item Built with \href{https://kotlinlang.org}{Kotlin}.
			\end{itemize}
		}
	}
	\hypertarget{scr}{
		\cventry{April 2019 \\ May 2022}{scr-cli \& scr-web}{\ghurl{AppleGamer22/scr-cli} \& \ghurl{AppleGamer22/scr-web}}{}{}{
			\begin{itemize}
				\item My previous attempt at building a full-stack (and a CLI) social media scraper with a single-page website framework and a RESTful server.
				\item Server-side is built with \href{https://typescriptlang.org}{TypeScript} \& \href{https://nestjs.com}{Nest} (with a \href{https://nodejs.org}{Node.js} runtime) \href{https://www.mongodb.com}{MongoDB}, \href{https://jwt.io}{JSON Web Tokens} (JWTs) and \href{https://www.docker.com}{Docker}.
				\item Client-side is built with \href{https://angular.io}{Angular} and \href{https://ionicframework.com}{Ionic}.
				\item The full-stack packages is bundled with \href{https://nx.dev}{Nx}.
				\item The CLI is built with \href{https://oclif.io}{OCLIF}
			\end{itemize}
		}
	}

	\subsection{Research}%\label{sec:research}
	\hypertarget{FIT2082}{
		\cventry{August 2021 \\ December 2021}{Software Contributor}{Monash University's \href{https://handbook.monash.edu/2021/units/FIT2082}{FIT2082 unit}}{Melbourne}{}{
			\begin{itemize}
				\item I \href{https://github.com/AppleGamer22/FIT2082}{contributed} to an \href{https://bitbucket.org/gkgange/lazycbs}{existing codebase}, based on prior research by \href{https://ojs.aaai.org/index.php/ICAPS/article/view/3471}{(Gange, Harabor and Stuckey, 2021)} about \textsl{Lazy CBS}, their \href{https://en.wikipedia.org/wiki/Pathfinding\#Multi-agent_pathfinding}{Multi-Agent Path Finding} (MAPF) algorithm.
				\begin{itemize}
					\item I modified the \textsl{Lazy CBS} codebase such that the algorithm also outputs the final set of constraints that is used to rule out paths, such that \textsl{Lazy CBS} is formally an \textbf{Explainable} Multi-Agent Path Finding (XMAPF) algorithm.
					\item I learned how to enable \httplink[\textsl{Python}-to-\textsl{C++} bindings]{pybind11.readthedocs.io/en/stable/}, such that the compiled \textsl{Lazy CBS} codebase can be used as a Python-facing library for future projects.
				\end{itemize}
				\item Built with C/C++ and \href{https://python.org}{Python} on top of Linux.
			\end{itemize}
		}
	}

	\section{Experience}
	\subsection{Clubs \& Societies}
	\cventry{June 2022 \\ Present}{Vice President}{\href{https://monsec.io/team/}{Monash University's Cyber Security Club} (MonSec)}{Melbourne}{}{\begin{itemize}
		\item Coordinated collaboration with the university's \href{https://www.monash.edu/it}{Faculty of Information Technology} for purposes of events and advertising.
		\item Club \href{https://monsec.io}{website}:
		\begin{itemize}
			\item Updated the \href{https://github.com/panr/hugo-theme-terminal}{theme} to its latest version, and resolved new layout bugs in collaboration with other club committee members.
			\item Improved the \href{https://kali.org}{Kali Linux} virtual machine \href{https://monsec.io/resources/\#kali-linux-set-up}{set-up guide} such that it includes more details on alternative installation methods.
		\end{itemize}
		\item Changed our \href{https://ctf.monsec.io}{CTFd}'s \href{https://github.com/hmrserver/CTFd-theme-pixo}{theme}. In addition, I disabled the theme's default CRT-like flicker effect by editing its CSS files, such that people with epilepsy would not get an inferior experience.
		\item Club representation:
		\begin{itemize}
			\item Faculty of IT's \href{https://www.monash.edu/it/events/2022/take-ctrl}{Take CTRL} (Cryptography \& Web Hacking Workshop)
			\item Faculty of IT's \href{https://www.monash.edu/it/events/2022/munch-and-mingle}{Munch \& Mingle}
			\item Faculty of IT's open day
			\item Orientation week of 2022's $2^{\text{nd}}$ semester.
		\end{itemize}
		\item \href{https://en.wikipedia.org/wiki/Capture_the_flag_(cybersecurity)}{Capture the Flag} (CTF) participation:
		\begin{itemize}
			\item \href{https://applegamer22.github.io/posts/uactf/}{The University of Adelaide's CTF}
			\item \href{https://applegamer22.github.io/posts/shell_ctf/}{SHELL CTF}
			\item \href{https://applegamer22.github.io/posts/downunderctf/}{DownUnderCTF}
		\end{itemize}
	\end{itemize}}
	\cventry{January 2022 \\ June 2022}{Secretary}{\href{https://monsec.io/team/}{Monash University's Cyber Security Club} (MonSec)}{Melbourne}{}{\begin{itemize}
		\item Organised and recorded official committee and club meetings.
		\item Represented the club during the orientation week of 2022's $1^{\text{st}}$ semester.
		\item Organised and ran a binary-level reverse engineering workshop (a recording is available on \href{https://youtu.be/893L13SxDUg}{YouTube}).
		\item Started a section on the \href{https://monsec.io/resources/}{resources page} of the club's website, with a detailed section with a guide on how to easily install and set-up a \href{https://www.kali.org}{Kali Linux} virtual machine.
	\end{itemize}}
	\cventry{May 2021 \\ January 2022}{General Representative}{\href{https://monsec.io/team/}{Monash University's Cyber Security Club} (MonSec)}{Melbourne}{}{\begin{itemize}
		\item Helped to organise and ran a workshop about brute-forcing tools used for penetration testing.
		\item Participated in \href{https://applegamer22.github.io/posts/angstromctf/}{{\aa}ngstromCTF}
	\end{itemize}}

	\subsection{Freelancing}
	\cventry{June 2021 \\ December 2021}{Software Engineer}{Contract}{Melbourne}{}{\begin{itemize}
		\item I implemented a fault-tolerant file back-up system that enables the continuation of file transferring from an variably-approximate point in time before the back-up disruption.
		\item Built with \href{https://go.dev}{Go}.
	\end{itemize}}
\end{document}
\documentclass[a4paper, 10pt]{article}
\usepackage[english]{babel}
\usepackage{fancyhdr}
\usepackage{hyperref}
\usepackage{titling}
\usepackage[empty]{fullpage}
\usepackage[none]{hyphenat}
\usepackage{fontspec}
\usepackage{xcolor}

\renewcommand{\headrulewidth}{0pt}
\renewcommand{\footrulewidth}{0pt}
\renewcommand{\labelitemii}{$\circ$}

\addtolength{\oddsidemargin}{-0.530in}
\addtolength{\evensidemargin}{-0.375in}
\addtolength{\textwidth}{1in}
\addtolength{\topmargin}{-0.45in}
\addtolength{\textheight}{1in}
\setlength{\headheight}{18pt}
\setlength{\footskip}{3.7pt}
\setlength{\tabcolsep}{0in}
\setlength{\parindent}{0pt}
\setcounter{secnumdepth}{-\maxdimen}
\setmonofont[Contextuals=Alternate, Scale=MatchLowercase]{Fira Code}

\author{Omri Bornstein}
\title{\theauthor's CV}
\date{\today}
\hypersetup{
	pdftitle={\thetitle},
	pdfsubject={\thetitle},
	pdfauthor={\theauthor},
	pdfkeywords={CV},
	colorlinks=true,
	linkcolor=purple,
	filecolor=purple,
	citecolor=blue,
	urlcolor=blue,
}
\urlstyle{same}
\pagestyle{fancy}
\fancyhf{}
\lhead{\href{mailto:omribor@gmail.com}{omribor@gmail.com}}
\chead{\Large \theauthor}
\rhead{\href{https://applegamer22.github.io}{applegamer22.github.io}}

\raggedleft\raggedright\begin{document}
	\section{Education}
	\begin{enumerate}
		\item \textbf{2017 --- 2019}: secondary education at \href{https://asms.sa.edu.au}{Australian Science \& Mathematics School} (ASMS), Adelaide, South Australia
		\item \textbf{2020 --- Present}: Bachelor of Computer Science at \href{https://www.monash.edu/it}{Monash University}, Melbourne, Victoria
	\end{enumerate}

	\section{Skills}
	\begin{itemize}
		\item \textbf{Computer Programming Languages}: \href{https://go.dev}{Go}, \href{https://typescriptlang.org}{TypeScript}/JavaScript, \href{https://python.org}{Python}, \href{https://kotlinlang.org}{Kotlin}/Java, C/C++
		\item \textbf{Document Markup Languages}: HTML/CSS, \TeX/\LaTeX, Markdown
		% \item \textbf{Natural Languages}: English, Hebrew
		\item \textbf{Databases}: \href{https://www.mongodb.com}{MongoDB}
		\item \textbf{Tools}: \href{https://git-scm.com}{Git}, \href{https://github.com}{GitHub}/\href{https://gitlab.com}{GitLab}, \href{https://www.docker.com}{Docker}, \href{https://kubernetes.io}{Kubernetes}, CI/CD
		\item \textbf{Platforms}: Linux, Cloud Native, web servers/browsers, macOS, Windows
		\item \textbf{Soft Skills}: technical writing, presenting/public speaking, research, troubleshooting, explaining, collaboration/teamwork
	\end{itemize}

	\section{Leadership Experience}
	\begin{enumerate}
		\item \textbf{May 2021 --- January 2022}: General Representative at \href{https://monsec.io/team/}{Monash University's Cyber Security Club} (MonSec)
		\item \textbf{January 2022 --- June 2022}: Secretary at \href{https://monsec.io/team/}{Monash University's Cyber Security Club} (MonSec)
		\item \textbf{June 2022 --- Present}: Vice President at \href{https://monsec.io/team/}{Monash University's Cyber Security Club} (MonSec)
	\end{enumerate}

	\section{Projects}
	\subsection{Open-Source}
	\begin{itemize}
		\item \href{https://github.com/AppleGamer22/cocainate}{\texttt{cocainate}} is a cross-platform re-implementation of the macOS utility \href{https://github.com/apple-oss-distributions/PowerManagement/tree/main/caffeinate}{\texttt{caffeinate}} that keeps the screen turned on either until stopped, for a set duration of time or while another process still runs.
		\item \href{https://github.com/AppleGamer22/stalk}{\texttt{stalk}} is a cross-platform file-watcher that can run a command after each file-system operation on a given files or simply wait once until a file is changed.
		\item \href{https://github.com/AppleGamer22/raker}{\texttt{raker}} is a social media scraper that is interfaced via a server-side rendered HTML user interface (or a CLI), and is managed by a REST API and a NoSQL database.
		\item \href{https://github.com/AppleGamer22/scr-web}{scr-web} (and its \href{https://github.com/AppleGamer22/scr-cli}{scr-cli} counterpart) is my previous attempt at building a full-stack social media scraper with \href{https://angular.io}{Angular} on the front-end, and \href{https://nestjs.com}{Nest} on the back-end.
		\item \href{https://github.com/AppleGamer22/sp}{sp} is my first attempt at building a Minecraft server plugin. This plugin adds the requirement that the player supplies the password (via a server command) before proper server interaction is allowed, and as long as the password isn't provided, the currently-unauthorized player is blinded and immobile.
		% \item \href{https://github.com/monsec/ctftime-discord-bot}{ctftime-discord-bot} is a discord bot I made for the \href{https://monsec.io}{MonSec} Discord server, that fetches statistics about competing teams from \href{https://ctftime.org}{CTFtime}, and displays them in the Discord interface.
	\end{itemize}

	\subsection{Research}
	\begin{itemize}
		\item As part of the \href{https://handbook.monash.edu/2021/units/FIT2082}{FIT2082 unit}, I \href{https://github.com/AppleGamer22/FIT2082}{contributed} to an \href{https://bitbucket.org/gkgange/lazycbs}{existing codebase}, based on prior research by \href{https://ojs.aaai.org/index.php/ICAPS/article/view/3471}{(Gange, Harabor and Stuckey, 2021)} about \textsl{Lazy CBS}, their \href{https://en.wikipedia.org/wiki/Pathfinding#Multi-agent_pathfinding}{Multi-Agent Path Finding} (MAPF) algorithm.
		\begin{itemize}
			\item My task was to modify the \textsl{Lazy CBS} codebase such that the algorithm also outputs the final set of constraints that is used to rule out paths, such that \textsl{Lazy CBS} is formally an Explainable Multi-Agent Path Finding (XMAPF) algorithm.
			\item I learned how to enable \href{https://pybind11.readthedocs.io/en/stable/}{\textsl{Python}-to-\textsl{C++} bindings}, such that the compiled \textsl{Lazy CBS} codebase can be used as a Python-facing library for future projects.
		\end{itemize}
	\end{itemize}

	\subsection{Freelancing}
	\begin{itemize}
		\item I implemented a fault-tolerant file back-up system that enables the continuation of file transferring from an variably-approximate point in time before the back-up disruption.
	\end{itemize}
\end{document}
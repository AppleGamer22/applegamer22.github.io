\documentclass[a4paper]{article}

\usepackage[english]{babel}
\usepackage{fancyhdr}
\usepackage[colorlinks=true, allcolors=blue]{hyperref}
\usepackage[empty]{fullpage}
\usepackage{titlesec}

\renewcommand{\headrulewidth}{0pt}
\renewcommand{\footrulewidth}{0pt}
\addtolength{\oddsidemargin}{-0.530in}
\addtolength{\evensidemargin}{-0.375in}
\addtolength{\textwidth}{1in}
\addtolength{\topmargin}{-0.45in}
\addtolength{\textheight}{1in}
\setlength{\headheight}{18pt}
\setlength{\footskip}{3.7pt}
\setlength{\tabcolsep}{0in}
\urlstyle{rm}
% \raggedbottom
% \raggedright

\title{Omri Bornstein's CV}
\author{Omri Bornstein}
\pagestyle{fancy}
\fancyhf{}
\lhead{\href{mailto:omribor@gmail.com}{omribor@gmail.com}}
\chead{\Large Omri Bornstein}
\rhead{\href{https://applegamer22.github.io}{applegamer22.github.io}}

\begin{document}
	\section*{Education}

	\begin{enumerate}
		\item \textbf{2017 - 2019}: secondary education at Australian Science \& Mathematics School, Adelaide, South Australia
		\item \textbf{2020 - Present}: Bachelor of Computer Science at Monash University, Melbourne, Victoria
	\end{enumerate}

	\section*{Skills}

	\begin{itemize}
		\item \textbf{Computer Programming Languages}: Go, TypeScript/JavaScript, Pyhton, Kotlin/Java, C/C++
		\item \textbf{Document Markup Languages}: HTML, CSS, \TeX, Markdown
		\item \textbf{Natural Languages}: English, Hebrew
		\item \textbf{Tools}: Git, GitHub, Docker, Kubernetes, GoReleaser, GitHub Actions
		\item \textbf{Platforms}: Linux, Cloud Native, Web
		\item \textbf{Soft Skills}: Writing, Presenting, Collaboration
	\end{itemize}

	\section*{Experience}

	\begin{enumerate}
		\item \textbf{May 2021 - January 2022}: General Representative at Monash Cyber Security Club.
		\item \textbf{January 2022 - June 2022}: Secretary at Monash Cyber Security Club.
		\item \textbf{June 2022 - Present}: Vice President at Monash Cyber Security Club.
	\end{enumerate}

	\section*{Projects}
	\subsection*{Open-Source}

	\begin{itemize}
		\item \href{https://github.com/AppleGamer22/cocainate}{\texttt{cocainate}} is a cross-platform re-implementation of the macOS utility \href{https://github.com/apple-oss-distributions/PowerManagement/tree/main/caffeinate}{\texttt{caffeinate}} that keeps the screen turned on either until stopped, for a set duration of time or while another process still runs.
		\item \href{https://github.com/AppleGamer22/stalk}{\texttt{stalk}} is a cross-platform file-watcher that can run a command after each file-system operation on a given file(s) or simply wait once until a file is changed.
		\item \href{https://github.com/AppleGamer22/rake}{\texttt{rake}} is a social media scraper that is interfaced via a server-side rendered HTML user interface (or a CLI), and is managed by a REST API and a NoSQL database.
		\item \href{https://github.com/AppleGamer22/scr-web}{scr-web} (and its \href{https://github.com/AppleGamer22/scr-cli}{scr-cli} counterpart) is my previous attempt at building a full-stack social media scraper, which was abandoned due to the excessive number of dependencies and the rather large build-size.
		\item \href{https://github.com/AppleGamer22/sp}{sp} is my first attempt at building a Minecraft server plugin. This plugin adds the requirement that the player supplies the password (via a server command) before proper server interaction is allowed. Until as password is provided, the currently-unauthorized player is blinded and immobile.
	\end{itemize}

	\subsection*{Research}
	As part of the \href{https://handbook.monash.edu/2021/units/FIT2082}{FIT2082 unit}, I contributed to an existing codebase, based on prior research by \href{https://ojs.aaai.org/index.php/ICAPS/article/view/3471}{(Gange, Harabor and Stuckey, 2021)} about Lazy CBS, their Multi-Agent Path Finding (MAPF) algorithm. The MAPF problem is a subset of the path finding research field, which presents the additional requirements of multiple agents, each with a unique pair of a source and a target, such that the path between them does not intersect with another path during the same point in time. My task was to modify the Lazy CBS codebase such that the algorithm also outputs the final set of constraints that is used to rule out possible paths, such that the Lazy is formally an Explainable Multi-Agent Path Finding (XMAPF) algorithm. In addition, I added Python-to-C++ bindings, such that the compiled Lazy CBS codebase can be used as a Python-facing library for future projects.

	% \subsection*{Freelancing}
\end{document}
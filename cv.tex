\documentclass[a4paper, 10pt, roman, colorlinks, linkcolor=purple, filecolor=purple, citecolor=blue, urlcolor=blue]{moderncv}
\moderncvstyle{classic}

\usepackage[margin=15mm]{geometry}
\usepackage[utf8]{inputenc}
\usepackage[english]{babel}
\usepackage[none]{hyphenat}
\usepackage{amsmath}
% \usepackage{fontspec}

\renewcommand{\labelitemi}{$\bullet$}
\renewcommand{\labelitemii}{$\circ$}

% \setmonofont[Contextuals=Alternate, Scale=MatchLowercase]{Fira Code}
% \urlstyle{same}
\setlength{\hintscolumnwidth}{0.15\textwidth}

\name{Omri}{Bornstein}
\title{Software Engineer}
\address{}{Greater Melbourne Area}{Australia}
% \phone{+61 448 289 895}
\email{omribor@gmail.com}
\homepage{applegamer22.github.io}
\social[linkedin]{omri-bornstein}
\social[github]{AppleGamer22}
% \social[twitter]{AppleGamer22}
\date{\today}

\raggedleft\raggedright\begin{document}
	\maketitle
	\section{Education}
	\cventry{2017 \\ 2019}{South Australian Certificate of Education}{\href{https://asms.sa.edu.au}{Australian Science \& Mathematics School} (ASMS)}{Adelaide}{}{}
	\cventry{2020 \\ Present}{Bachelor of Computer Science}{\href{https://www.monash.edu/it}{Monash University}}{Melbourne}{}{}

	\section{Skills}
	% \cvlanguage{name}{level}{comment}
	\begin{itemize}
		\item \textbf{Computer Programming Languages}: \href{https://go.dev}{Go}, \href{https://typescriptlang.org}{TypeScript}/JavaScript, \href{https://python.org}{Python}, \href{https://kotlinlang.org}{Kotlin}/Java, C/C++
		\item \textbf{Document Markup Languages}: HTML/CSS, \TeX/\LaTeX, Markdown
		% \item \textbf{Natural Languages}: English, Hebrew
		\item \textbf{Databases}: \href{https://www.mongodb.com}{MongoDB}
		\item \textbf{Tools}: \href{https://git-scm.com}{Git}, \href{https://github.com}{GitHub}/\href{https://gitlab.com}{GitLab}, \href{https://www.docker.com}{Docker}, \href{https://kubernetes.io}{Kubernetes}, CI/CD
		\item \textbf{Platforms}: Linux, Cloud Native, web servers/browsers, macOS, Windows
		\item \textbf{Soft Skills}: technical writing, presenting/public speaking, research, troubleshooting/debugging, explaining, collaboration/teamwork
	\end{itemize}

	\section{Leadership Experience}
	\cventry{May 2021 \\ January 2022}{General Representative}{\href{https://monsec.io/team/}{Monash University's Cyber Security Club} (MonSec)}{Melbourne}{}{\begin{itemize}
		\item Helped to organise and ran a workshop about brute-forcing tools used for penetration testing.
		\item Participated in \href{https://applegamer22.github.io/posts/angstromctf/}{{\aa}ngstromCTF}
	\end{itemize}}
	\cventry{January 2022 \\ June 2022}{Secretary}{\href{https://monsec.io/team/}{Monash University's Cyber Security Club} (MonSec)}{Melbourne}{}{\begin{itemize}
		\item Organised and recorded official committee and club meetings.
		\item Represented the club during the orientation week of 2022 1$^{\text{st}}$ semester.
		\item Organised and ran a binary-level reverse engineering workshop (a recording is available available at \url{https://youtu.be/893L13SxDUg}).
		\item Started an expanded \href{https://monsec.io/resources/}{resources page} on the club's website, with a detailed section with a guide on how to easily install and set-up a \href{https://www.kali.org}{Kali Linux} virtual machine.
	\end{itemize}}
	\cventry{June 2022 \\ Present}{Vice President}{\href{https://monsec.io/team/}{Monash University's Cyber Security Club} (MonSec)}{Melbourne}{}{\begin{itemize}
		\item Coordinated collaboration with the university's \href{https://www.monash.edu/it}{Faculty of Information Technology} for purposes of events and advertising.
		\item Updated the \href{https://monsec.io}{website}'s \href{https://github.com/panr/hugo-theme-terminal}{theme} to its latest version, and resolved new layout bugs in collaboration with other club committee members.
		\item Club representation:
		\begin{itemize}
			\item Faculty of IT open day
			\item Orientation week of 2022's 2$^{\text{nd}}$ semester.
		\end{itemize}
		\item \href{https://en.wikipedia.org/wiki/Capture_the_flag_(cybersecurity)}{Capture the Flag} (CTF) participation:
		\begin{itemize}
			\item \href{https://applegamer22.github.io/posts/uactf/}{The University of Adelaide's CTF}
			\item \href{https://applegamer22.github.io/posts/shell_ctf/}{SHELL CTF}
		\end{itemize}
	\end{itemize}}

	\section{Projects}
	\subsection{Open-Source}
	\cventry{January 2022 \\ Present}{\texttt{cocainate}}{\url{https://github.com/AppleGamer22/cocainate}}{}{}{\begin{itemize}
		\item A cross-platform re-implementation of the macOS utility \href{https://github.com/apple-oss-distributions/PowerManagement/tree/main/caffeinate}{\texttt{caffeinate}} that keeps the screen turned on either until stopped, for a set duration of time or while another process still runs.
		\item Built with \href{https://go.dev}{Go} and \href{https://cobra.dev}{Cobra}.
	\end{itemize}}
	\cventry{May 2022 \\ Present}{\texttt{stalk}}{\url{https://github.com/AppleGamer22/stalk}}{}{}{\begin{itemize}
		\item A cross-platform file-watcher that can run a command after each file-system operation on a given files or simply wait once until a file is changed.
		\item Built with \href{https://go.dev}{Go}, \href{https://cobra.dev}{Cobra} and \href{https://github.com/fsnotify/fsnotify}{FSnotify}.
	\end{itemize}}
	\cventry{May 2022 \\ Present}{\texttt{raker}}{\url{https://github.com/AppleGamer22/raker}}{}{}{\begin{itemize}
		\item A social media scraper that is interfaced via a server-side rendered HTML user interface (or a CLI), and is managed by a REST API and a NoSQL database.
		\item Server-side is built with:
		\begin{itemize}
			\item \href{https://go.dev}{Go}
			\item \href{https://www.mongodb.com}{MongoDB}
			\item \href{https://jwt.io}{JSON Web Tokens}
			\item \href{https://www.docker.com}{Docker}
		\end{itemize}
		\item Client-side is built with HTML/CSS (\href{https://getbootstrap.com}{Bootstrap}).
		\item The companion CLI utility and configuration are built with \href{https://cobra.dev}{Cobra} and \href{https://github.com/spf13/viper}{Viper}.
	\end{itemize}}
	\cventry{December 2021}{CTFtime Discord Bot}{\url{https://github.com/monsec/ctftime-discord-bot}}{}{}{\begin{itemize}
		\item A Discord bot for \href{https://monsec.io/contact/}{MonSec}'s Discord server, that fetches statistics about competing \href{https://en.wikipedia.org/wiki/Capture_the_flag_(cybersecurity)}{Capture the Flag} (CTF) teams from \href{https://ctftime.org}{CTFtime}, and displays them in the Discord interface.
		\item Built with \href{https://go.dev}{Go}.
	\end{itemize}}
	\cventry{June 2020 \\ January 2021}{sp}{\url{https://github.com/AppleGamer22/sp}}{}{}{\begin{itemize}
		\item My first attempt at building a \href{https://papermc.io}{Minecraft server plugin}. This plugin adds the requirement that the player supplies the password (via a server command) before proper server interaction is allowed, and as long as the password isn't provided, the currently-unauthorized player is blinded and immobile.
		\item Built with \href{https://kotlinlang.org}{Kotlin}.
	\end{itemize}}
	\cventry{April 2019 \\ May 2022}{scr-cli \& scr-web}{\url{https://github.com/AppleGamer22/scr-cli} \& \url{https://github.com/AppleGamer22/scr-web}}{}{}{\begin{itemize}
		\item My previous attempt at building a full-stack (and a CLI) social media scraper with a single-page website framework and a RESTful server.
		\item Server-side is built with:
		\begin{itemize}
			\item \href{https://typescriptlang.org}{TypeScript} \& \href{https://nestjs.com}{Nest} (with a \href{https://nodejs.org}{Node.js} runtime)
			\item \href{https://www.mongodb.com}{MongoDB}
			\item \href{https://jwt.io}{JSON Web Tokens}
			\item \href{https://www.docker.com}{Docker}
		\end{itemize}
		\item Client-side is built with:
		\begin{itemize}
			\item \href{https://angular.io}{Angular}
			\item \href{https://ionicframework.com}{Ionic}
		\end{itemize}
		\item The full-stack packages is bundled with \href{https://nx.dev}{Nx}.
		\item The CLI is built with \href{https://oclif.io}{OCLIF}
	\end{itemize}}

	\subsection{Research}
	\cventry{August 2021 \\ December 2021}{Software Contributor}{Monash University's \href{https://handbook.monash.edu/2021/units/FIT2082}{FIT2082 unit}}{Melbourne}{}{\begin{itemize}
		\item I \href{https://github.com/AppleGamer22/FIT2082}{contributed} to an \href{https://bitbucket.org/gkgange/lazycbs}{existing codebase}, based on prior research by \href{https://ojs.aaai.org/index.php/ICAPS/article/view/3471}{(Gange, Harabor and Stuckey, 2021)} about \textsl{Lazy CBS}, their \href{https://en.wikipedia.org/wiki/Pathfinding\#Multi-agent_pathfinding}{Multi-Agent Path Finding} (MAPF) algorithm.
		\begin{itemize}
			\item I modified the \textsl{Lazy CBS} codebase such that the algorithm also outputs the final set of constraints that is used to rule out paths, such that \textsl{Lazy CBS} is formally an \textbf{Explainable} Multi-Agent Path Finding (XMAPF) algorithm.
			\item I learned how to enable \httplink[\textsl{Python}-to-\textsl{C++} bindings]{pybind11.readthedocs.io/en/stable/}, such that the compiled \textsl{Lazy CBS} codebase can be used as a Python-facing library for future projects.
		\end{itemize}
		\item Built with C/C++ and \href{https://python.org}{Python} on top of Linux.
	\end{itemize}}

	\subsection{Freelancing}
	\cventry{June 2021 \\ December 2021}{Software Engineer}{Contract}{Melbourne}{}{\begin{itemize}
		\item I implemented a fault-tolerant file back-up system that enables the continuation of file transferring from an variably-approximate point in time before the back-up disruption.
		\item Built with \href{https://go.dev}{Go}.
	\end{itemize}}
\end{document}